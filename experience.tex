%START ASCII
\begin{rSubsection}{Qualcomm Technologies, Inc.}{Jan 2009 - Present}{Sr. Engineer}{}
  \item[] Technology lead for the inter-processor communication stack encompassing all layers from session to link layer used for messaging within the Qualcomm Snapdragon SoC and all supported operating systems (Linux, QNX, L4 \& QuRT - A Qualcomm proprietary RTOS).
  \item[]
  \item[] \textbf{Accomplishments}
  \begin{itemize}
    \item Designed and developed an inter-processor communication routing protocol for the embedded system supporting dynamic routing, name service, unicast and multicast messaging. The module currently supports USB, PCIE \& shared memory.
    \begin{itemize}
      \item[--] Improved the throughput of the routing protocol over PCIE by a factor of 200\%.
      \item[--] Reduced code size by a factor of 50\% to enable support of devices with lower capabilities.
      \item[--] Developed a socket interface to facilitate easy adoption.
    \end{itemize}
    \item Implemented a client/server interface (providing the presentation and session layers of the OSI stack) allowing users to exchange structured messages seamlessly within a heterogeneous environment such as the Snapdragon SoC. This was later extended to support one to many (multicast) messaging.
    \item Key developer of a generic link layer protocol capable of adapting to a multitude of physical transports from wired to zero-copy.
    \begin{itemize}
     \item Developed transport plugins for SPI, UART \& SMEM.
     \item Worked with the audio team in improving the worst case IPC latency by a factor of 4 on Linux.
     \item Ported all drivers to the QNX (Auto), freeRTOS \& ThreadX platforms and successfully delivered on an aggressive deadline
    \end{itemize}
    \item Work closely with application and hardware teams from design to final implementation. Key accomplishments being the creation of a unified sensor messaging platform, reduce platform dependence in the radio interface layer.
    \item Developed a multi-processor simulation platform to test IPC device drivers off-target enabling continuous integration testing which allowed for the rapid development and testing of features, bug-fixes and eventually drive product quality.
  \end{itemize}
\end{rSubsection}

%------------------------------------------------

\begin{rSubsection}{VMware, Inc.}{May 2008 - Aug 2008}{Kernel Intern}{}
%  \begin{itemize}
  \item[] Investigated nested virtual machine creation and their uses in test infrastructure and automation.
  \item[] Developed a mechanism to create a core dump of the guest VM in runaway and locked up virtual machines.
%  \end{itemize}
\end{rSubsection}

%------------------------------------------------

%\begin{rSubsection}{Caritor India Pvt. Ltd.}{Sep 2005 - Jul 2006}{Software Engineer}{}
%  \item Responsibilities included creation and maintenance of test cases and test models.
%\end{rSubsection}
%END ASCII
