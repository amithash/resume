%START ASCII

\begin{description}
\item[Thermal Impact of Compiler optimization and Co-Scheduling on a CMP Architecture]
\textit{Computer Performance Modeling, Spring '08} - In-depth statistical study of the thermal profile of applications and their distortion due to compiler optimization and operating system co-scheduling choices on a multicore architecture. A linux kernel module was developed to gather time series data such as temperature, instructions per clock (IPC) etc. for the Intel Xeon processor (Penryn architecture)

\item[Parallelization of the K-Means Algorithm]
\textit{High Performance Scientific Computing, Fall '07} - The K-means clustering algorithm was parallelized using the Message passing interface and its scalability was studied on a IBM Blue Gene high performance machine

\item[Adaptive Noise Cancellation, A hardware and software approach]
\textit{Hybrid Embedded Systems, Fall '07} - The LMS (Least means square) filter for adaptive noise cancellation was built both in C and VHDL on the Xilinx Virtex II development board and the performance gain for the hardware implementation was found to be 5-6 times faster.

\item[Cache Coherence Phase analysis and effects of Process Migration]
\textit{Parallel Processing, Spring '07} - Cache coherence phase state was profiled by sampling the AMD Opteron hardware performance counters, and the effects of process migration was observed for various state transitions. A linux kernel module was developed to aid in the time series data collection.

\item[Evolution]
\textit{Compiler Code Generation and Optimization, Spring '07} - Developed an iterative compiler front-end for GCC written in Perl to generate various binary versions which are compiled using randomized sets of compiler optimizations.

\item[Value profiling and prediction]
\textit{Advanced Computer Architecture, Fall '06} - Study of register value invariance profiling, simulation and analysis of various value prediction schemes and their performance dependence on compiler optimization. PIN was used as an architecture instrumentation tool for the project.

\item[Hybrid gAPNADALINE+PAg Branch Predictor]
\textit{Advanced Computer Architecture, Fall '06} - Simulation of a branch predictor using PIN instrumentation tool which uses a hybrid mechanism combining an ADALINE neural network and a PAG Scheme.

\item[Human controlled remote robotic arm]
\textit{Real-Time Embedded Systems, Fall '06} - A real time system to control a robotic arm using a glove interface on the VxWorks operating system.
\end{description}

%END ASCII
